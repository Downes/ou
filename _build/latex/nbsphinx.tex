%% Generated by Sphinx.
\def\sphinxdocclass{report}
\documentclass[letterpaper,10pt,english]{sphinxmanual}
\ifdefined\pdfpxdimen
   \let\sphinxpxdimen\pdfpxdimen\else\newdimen\sphinxpxdimen
\fi \sphinxpxdimen=.75bp\relax

\PassOptionsToPackage{warn}{textcomp}
\usepackage[utf8]{inputenc}
\ifdefined\DeclareUnicodeCharacter
% support both utf8 and utf8x syntaxes
  \ifdefined\DeclareUnicodeCharacterAsOptional
    \def\sphinxDUC#1{\DeclareUnicodeCharacter{"#1}}
  \else
    \let\sphinxDUC\DeclareUnicodeCharacter
  \fi
  \sphinxDUC{00A0}{\nobreakspace}
  \sphinxDUC{2500}{\sphinxunichar{2500}}
  \sphinxDUC{2502}{\sphinxunichar{2502}}
  \sphinxDUC{2514}{\sphinxunichar{2514}}
  \sphinxDUC{251C}{\sphinxunichar{251C}}
  \sphinxDUC{2572}{\textbackslash}
\fi
\usepackage{cmap}
\usepackage[T1]{fontenc}
\usepackage{amsmath,amssymb,amstext}
\usepackage{babel}



\usepackage{times}
\expandafter\ifx\csname T@LGR\endcsname\relax
\else
% LGR was declared as font encoding
  \substitutefont{LGR}{\rmdefault}{cmr}
  \substitutefont{LGR}{\sfdefault}{cmss}
  \substitutefont{LGR}{\ttdefault}{cmtt}
\fi
\expandafter\ifx\csname T@X2\endcsname\relax
  \expandafter\ifx\csname T@T2A\endcsname\relax
  \else
  % T2A was declared as font encoding
    \substitutefont{T2A}{\rmdefault}{cmr}
    \substitutefont{T2A}{\sfdefault}{cmss}
    \substitutefont{T2A}{\ttdefault}{cmtt}
  \fi
\else
% X2 was declared as font encoding
  \substitutefont{X2}{\rmdefault}{cmr}
  \substitutefont{X2}{\sfdefault}{cmss}
  \substitutefont{X2}{\ttdefault}{cmtt}
\fi


\usepackage[Bjarne]{fncychap}
\usepackage{sphinx}

\fvset{fontsize=\small}
\usepackage{geometry}


% Include hyperref last.
\usepackage{hyperref}
% Fix anchor placement for figures with captions.
\usepackage{hypcap}% it must be loaded after hyperref.
% Set up styles of URL: it should be placed after hyperref.
\urlstyle{same}

\usepackage{sphinxmessages}
\setcounter{tocdepth}{1}


% Jupyter Notebook code cell colors
\definecolor{nbsphinxin}{HTML}{307FC1}
\definecolor{nbsphinxout}{HTML}{BF5B3D}
\definecolor{nbsphinx-code-bg}{HTML}{F5F5F5}
\definecolor{nbsphinx-code-border}{HTML}{E0E0E0}
\definecolor{nbsphinx-stderr}{HTML}{FFDDDD}
% ANSI colors for output streams and traceback highlighting
\definecolor{ansi-black}{HTML}{3E424D}
\definecolor{ansi-black-intense}{HTML}{282C36}
\definecolor{ansi-red}{HTML}{E75C58}
\definecolor{ansi-red-intense}{HTML}{B22B31}
\definecolor{ansi-green}{HTML}{00A250}
\definecolor{ansi-green-intense}{HTML}{007427}
\definecolor{ansi-yellow}{HTML}{DDB62B}
\definecolor{ansi-yellow-intense}{HTML}{B27D12}
\definecolor{ansi-blue}{HTML}{208FFB}
\definecolor{ansi-blue-intense}{HTML}{0065CA}
\definecolor{ansi-magenta}{HTML}{D160C4}
\definecolor{ansi-magenta-intense}{HTML}{A03196}
\definecolor{ansi-cyan}{HTML}{60C6C8}
\definecolor{ansi-cyan-intense}{HTML}{258F8F}
\definecolor{ansi-white}{HTML}{C5C1B4}
\definecolor{ansi-white-intense}{HTML}{A1A6B2}
\definecolor{ansi-default-inverse-fg}{HTML}{FFFFFF}
\definecolor{ansi-default-inverse-bg}{HTML}{000000}

% Define an environment for non-plain-text code cell outputs (e.g. images)
\makeatletter
\newenvironment{nbsphinxfancyoutput}{%
    % Avoid fatal error with framed.sty if graphics too long to fit on one page
    \let\sphinxincludegraphics\nbsphinxincludegraphics
    \nbsphinx@image@maxheight\textheight
    \advance\nbsphinx@image@maxheight -2\fboxsep   % default \fboxsep 3pt
    \advance\nbsphinx@image@maxheight -2\fboxrule  % default \fboxrule 0.4pt
    \advance\nbsphinx@image@maxheight -\baselineskip
\def\nbsphinxfcolorbox{\spx@fcolorbox{nbsphinx-code-border}{white}}%
\def\FrameCommand{\nbsphinxfcolorbox\nbsphinxfancyaddprompt\@empty}%
\def\FirstFrameCommand{\nbsphinxfcolorbox\nbsphinxfancyaddprompt\sphinxVerbatim@Continues}%
\def\MidFrameCommand{\nbsphinxfcolorbox\sphinxVerbatim@Continued\sphinxVerbatim@Continues}%
\def\LastFrameCommand{\nbsphinxfcolorbox\sphinxVerbatim@Continued\@empty}%
\MakeFramed{\advance\hsize-\width\@totalleftmargin\z@\linewidth\hsize\@setminipage}%
\lineskip=1ex\lineskiplimit=1ex\raggedright%
}{\par\unskip\@minipagefalse\endMakeFramed}
\makeatother
\newbox\nbsphinxpromptbox
\def\nbsphinxfancyaddprompt{\ifvoid\nbsphinxpromptbox\else
    \kern\fboxrule\kern\fboxsep
    \copy\nbsphinxpromptbox
    \kern-\ht\nbsphinxpromptbox\kern-\dp\nbsphinxpromptbox
    \kern-\fboxsep\kern-\fboxrule\nointerlineskip
    \fi}
\newlength\nbsphinxcodecellspacing
\setlength{\nbsphinxcodecellspacing}{0pt}

% Define support macros for attaching opening and closing lines to notebooks
\newsavebox\nbsphinxbox
\makeatletter
\newcommand{\nbsphinxstartnotebook}[1]{%
    \par
    % measure needed space
    \setbox\nbsphinxbox\vtop{{#1\par}}
    % reserve some space at bottom of page, else start new page
    \needspace{\dimexpr2.5\baselineskip+\ht\nbsphinxbox+\dp\nbsphinxbox}
    % mimick vertical spacing from \section command
      \addpenalty\@secpenalty
      \@tempskipa 3.5ex \@plus 1ex \@minus .2ex\relax
      \addvspace\@tempskipa
      {\Large\@tempskipa\baselineskip
             \advance\@tempskipa-\prevdepth
             \advance\@tempskipa-\ht\nbsphinxbox
             \ifdim\@tempskipa>\z@
               \vskip \@tempskipa
             \fi}
    \unvbox\nbsphinxbox
    % if notebook starts with a \section, prevent it from adding extra space
    \@nobreaktrue\everypar{\@nobreakfalse\everypar{}}%
    % compensate the parskip which will get inserted by next paragraph
    \nobreak\vskip-\parskip
    % do not break here
    \nobreak
}% end of \nbsphinxstartnotebook

\newcommand{\nbsphinxstopnotebook}[1]{%
    \par
    % measure needed space
    \setbox\nbsphinxbox\vbox{{#1\par}}
    \nobreak % it updates page totals
    \dimen@\pagegoal
    \advance\dimen@-\pagetotal \advance\dimen@-\pagedepth
    \advance\dimen@-\ht\nbsphinxbox \advance\dimen@-\dp\nbsphinxbox
    \ifdim\dimen@<\z@
      % little space left
      \unvbox\nbsphinxbox
      \kern-.8\baselineskip
      \nobreak\vskip\z@\@plus1fil
      \penalty100
      \vskip\z@\@plus-1fil
      \kern.8\baselineskip
    \else
      \unvbox\nbsphinxbox
    \fi
}% end of \nbsphinxstopnotebook

% Ensure height of an included graphics fits in nbsphinxfancyoutput frame
\newdimen\nbsphinx@image@maxheight % set in nbsphinxfancyoutput environment
\newcommand*{\nbsphinxincludegraphics}[2][]{%
    \gdef\spx@includegraphics@options{#1}%
    \setbox\spx@image@box\hbox{\includegraphics[#1,draft]{#2}}%
    \in@false
    \ifdim \wd\spx@image@box>\linewidth
      \g@addto@macro\spx@includegraphics@options{,width=\linewidth}%
      \in@true
    \fi
    % no rotation, no need to worry about depth
    \ifdim \ht\spx@image@box>\nbsphinx@image@maxheight
      \g@addto@macro\spx@includegraphics@options{,height=\nbsphinx@image@maxheight}%
      \in@true
    \fi
    \ifin@
      \g@addto@macro\spx@includegraphics@options{,keepaspectratio}%
    \fi
    \setbox\spx@image@box\box\voidb@x % clear memory
    \expandafter\includegraphics\expandafter[\spx@includegraphics@options]{#2}%
}% end of "\MakeFrame"-safe variant of \sphinxincludegraphics
\makeatother

\makeatletter
\renewcommand*\sphinx@verbatim@nolig@list{\do\'\do\`}
\begingroup
\catcode`'=\active
\let\nbsphinx@noligs\@noligs
\g@addto@macro\nbsphinx@noligs{\let'\PYGZsq}
\endgroup
\makeatother
\renewcommand*\sphinxbreaksbeforeactivelist{\do\<\do\"\do\'}
\renewcommand*\sphinxbreaksafteractivelist{\do\.\do\,\do\:\do\;\do\?\do\!\do\/\do\>\do\-}
\makeatletter
\fvset{codes*=\sphinxbreaksattexescapedchars\do\^\^\let\@noligs\nbsphinx@noligs}
\makeatother



\title{Climate change: island life in a volatile world}
\date{Mar 04, 2020}
\release{}
\author{}
\newcommand{\sphinxlogo}{\vbox{}}
\renewcommand{\releasename}{}
\makeindex
\begin{document}

\pagestyle{empty}
\sphinxmaketitle
\pagestyle{plain}
\sphinxtableofcontents
\pagestyle{normal}
\phantomsection\label{\detokenize{index::doc}}


Content generated from the OpenLearn Unit \sphinxhref{https://www.open.edu/openlearn/society-politics-law/sociology/climate-change-island-life-volatile-world/content-section-0}{Climate change: island life in a volatile world}.


\chapter{Contents:}
\label{\detokenize{index:contents}}

\section{Session 00}
\label{\detokenize{index:session-00}}

\subsection{1 Dividing the planet}
\label{\detokenize{content/session_00/Part_00_01:1-Dividing-the-planet}}\label{\detokenize{content/session_00/Part_00_01::doc}}
A good globe can set you back quite a lot of money. Of course, I don’t mean the little moulded plastic planets or the globes you can blow up as if the world were a beach ball, but the decent sized ones that sit solidly on turned wooden bases and quietly emanate authority from the corner of a room. Yet these days, it hardly seems worthwhile making such an investment. Countries appear to change their colour, their shape or their name with remarkable rapidity.

It has become a cliché to point out that globes and maps that date back to the middle of the last century and earlier featured vast swathes coloured in pink or red to signify lands that belonged to the British Empire. Not only has the British Empire broken up, but over recent decades we have also seen the dissolution of the Soviet Union into a number of new states. Other countries like Czechoslovakia, Yugoslavia and Ethiopia have split themselves into two or more pieces. Meanwhile, in the
western Pacific, a cluster of islands that was once a British colony called the Gilbert and Ellice Islands has become two republics: the Gilbert group are now known as Kiribati, while the Ellice Islands are now Tuvalu.

How we divide the planet’s surface up into recognisable units reveals ongoing changes: territorial reshufflings that may render a map or globe out of date before it has even left the production line. More importantly, such transformations are often contentious and painfully wrought at the ‘ground level’ where people live, frequently leaving some people unsettled or uncertain as to where they belong. But amid all this relentless activity, it can be tempting to think of the land masses themselves,
the continents and the islands, as maintaining the same shape, even as their names, colours or subdivisions change.

While there may be some comfort in the idea that land endures while all else changes, it is a rather dubious assumption. For several decades it has been generally accepted in the earth sciences that continents ‘drift’, usually at the rate of inches or fractions of an inch each year. More recently, a growing body of evidence suggests that much more rapid physical changes in the surface of our planet are also beginning to take place. Since the 1980s, intensive research by climate scientists
collaborating internationally has built up a picture of the earth’s weather systems being transformed by human activities. These changes are often referred to, in a kind of shorthand, as ‘global warming’.

The prospect of global climate change and what it might mean for the way we experience and imagine our planet is the theme of this course. As we will see, it is not easy to predict the extent or severity of future changes in the world’s climate, and it is just as difficult to anticipate how people, organisations and nation states will respond to these potentially changing conditions. Many climate scientists are now predicting that a generalised warming now under way will lead to gradually rising
sea levels throughout this century. This would impact on coastlines around the world, but it would have particularly serious implications for small islands or atolls.

Within the span of a single lifetime, islands which now support dense and vibrant populations could become too prone to climatic extremes to remain habitable. Some low\sphinxhyphen{}lying coral islands might even disappear completely beneath the surface of the sea.

\sphinxincludegraphics[width=443\sphinxpxdimen,height=227\sphinxpxdimen]{{dd205_3_001i}.jpg}

Figure 1 Boy with model outrigger canoe, Nukulaelae Atoll, Tuvalu


\subsubsection{Activity 1}
\label{\detokenize{content/session_00/Part_00_01:Activity-1}}

\paragraph{Question}
\label{\detokenize{content/session_00/Part_00_01:Question}}
Turn now to Reading 1A by Mark Lynas (2003) entitled ‘At the end of our weather’, which you will find attached below. Lynas, a journalist, felt moved by personal evidence of changing climate to seek out its impacts at ‘ground level’, and to give an impression of what it is like to live through these changes. While you are reading, you might want to pause and think what it would feel like to lose your whole country, never to be able to come back for a visit.
\begin{itemize}
\item {} 
Do you think you would try to stay even if weather events were potentially life\sphinxhyphen{}threatening?

\item {} 
If you chose to evacuate your country, where would you go?

\item {} 
What would you try to take with you?

\item {} 
Would you want all your friends and family, or all your compatriots to go to the same place?

\end{itemize}

Click to view Reading 1A. (2 pages, 0.08MB)

These are emotionally charged issues, and perhaps you are being asked to think about things which are so life\sphinxhyphen{}changing as to be almost unthinkable. However, these are also questions or dilemmas that some people, including the islanders of Tuvalu, have to live with in an everyday way. Steve Pile (2006) has written about disturbing events in the past that come back to haunt people in the present; the issue of climate change suggests that there are possible future events that can ‘haunt’ us here
and now

\sphinxincludegraphics[width=571\sphinxpxdimen,height=400\sphinxpxdimen]{{dd205_3_002i}.jpg}

Figure 2 Tuvalu’s location in the Pacific Ocean

In this way, the phenomenon of climate change, and the possibility of people displacement that it raises, impels us to think afresh about the ground beneath our feet, and the air and sea around us. It prompts us to question the permanence of what we may once have taken to be stable and enduring. The changes that are now being predicted are linked to patterns of energy use in the modern world. Accelerating industrial growth over the last two and a half centuries has relied predominantly on fossil
fuels that release carbon into the atmosphere, which scientific evidence suggests is contributing to an enhanced greenhouse effect that is warming the planet as a whole.


\subsubsection{Defining the enhanced greenhouse effect}
\label{\detokenize{content/session_00/Part_00_01:Defining-the-enhanced-greenhouse-effect}}
The greenhouse effect is a natural part of the functioning of the earth. It involves certain atmospheric gases (termed ‘greenhouse gases’) absorbing solar energy which has previously passed through the atmosphere and been reradiated back from the earth’s surface. This has the effect of keeping the planet many degrees warmer than would otherwise be expected from the amount of solar energy coming in. There is strong evidence that this natural greenhouse effect is now being enhanced as a result of
human activity. Burning fossil fuels and other activities change the composition of gases in the earth’s atmosphere \textendash{} adding significantly more greenhouse gases like carbon dioxide \textendash{} which results in an overall warming of the planet.

Recognising that industrial processes act cumulatively on climate focuses attention firmly on human activity as a potent force acting on the physical world. Indeed, some social theorists have argued that human\sphinxhyphen{}induced climate change, along with other damaging consequences of industrial activity on the environment, has now eclipsed natural disasters as a source of popular concern and anxiety (Beck, 1992).

However, amid this growing acknowledgement of the severity of human\sphinxhyphen{}induced or anthropogenic environmental issues, we should not forget that physical processes are highly variable, and sometimes extremely volatile, even without human input. Recent years have seen a number of sharp and shocking reminders of the forcefulness of the natural world, including earthquakes in Kobe, Japan (1997), Bam, Iran (2003) and the underwater quake off the island of Sumatra, Indonesia (2004) that triggered the
devastating tsunami in the Indian Ocean.

\sphinxincludegraphics[width=570\sphinxpxdimen,height=714\sphinxpxdimen]{{dd205_3_003i}.jpg}

Figure 3a and 3b Satellite images of the northern shore of Banda Aceh, Indonesia before and after the 2004 earthquake and tsunami

Click to view a larger version of Figures 3a and 3b.

Events on the scale of the Indonesian earthquake and tsunami can also alter the contours of land and sea \textendash{} even more rapidly than changes triggered by human activity. In the worst affected regions, whole towns and villages were destroyed and, even after the surges had receded, areas of coastline remained transformed. You may recall from news coverage that within days of the disaster in 2004, satellite photos taken before and after the event revealed sudden, dramatic changes, as can be seen in
Figures 3a and b. Within a few weeks, images from satellites and other sources had been compiled to produce a new atlas of the region which provided topographic information of the transformations as well as documenting the immediate social impacts of the disaster.

Human activities \textendash{} such as clearing away mangroves that once offered protection for coastlines \textendash{} may have contributed to the extent of the destruction caused by the Indian Ocean tsunami. Nonetheless, the earthquake itself, and the waves it triggered, was a natural event caused by shifts in the earth’s crust that remain beyond human influence.

This course starts by looking at the issue of human\sphinxhyphen{}induced change in global climate and its potential social impacts, but we will see that this issue soon draws our attention to ongoing changes in the world that are not directly attributable to human action. Concern with the human making and remaking of the world inevitably draws us to consider those other processes and events that have shaped our planet \textendash{} and will continue to shape it in the future.

As Lynas’s (2003) account (see Reading 1A) of the predicament of Tuvalu makes clear, the climate change issue raises questions about which particular groups or sectors of humanity have had the most impact, and which groups are most likely to suffer the worst consequences. Section 2 of this course looks at the way that climate change as a global process implicates people who are literally oceans apart. It introduces the notion of ‘territory’ as a way of coming to a clearer understanding of what
is under threat when we talk about serious changes in the world, and why people feel so strongly about threats to the places they live. Yet the very idea that one part of the globe can be affected by activities elsewhere on the planet also suggests that territories are connected in some way. This section also introduces the idea of ‘flows’ that move through and between territories, connecting them to the world beyond.

Thinking through territories and flows helps us to build up a sense of the different forces that come together to make and remake the world. In particular, it offers us a way of looking at both human and non\sphinxhyphen{}human forces, and how they work together. In Section 3, we will consider the long and rich story of human involvement in the making of islands, including the often awe\sphinxhyphen{}inspiring journeys that island settlers have undertaken to arrive at their new homelands far out in the ocean. Humans,
however, are not alone in settling islands, and they are rarely, if ever, the first to arrive. Therefore, in this section we will also examine the other forms of life that make their way to islands, and their contribution to island territories. Section 4 continues the theme of the importance of non\sphinxhyphen{}human forces in the shaping of islands, this time looking beyond the movements of human beings and other forms of life to the shifts and changes that take place in the earth itself and in the sea and
sky around us.

We delve beneath the issue of human\sphinxhyphen{}induced climate change, and its impact on low\sphinxhyphen{}lying islands like Tuvalu, in order to explore the long and profound entanglement of humans and non\sphinxhyphen{}human forces in the making of the world. Moreover, we begin to ponder what this entanglement might mean when addressing problems like climate change \textendash{} urgent and far\sphinxhyphen{}reaching problems that call out for some response.


\subsubsection{Course aims}
\label{\detokenize{content/session_00/Part_00_01:Course-aims}}\begin{itemize}
\item {} 
To explore how the world around us shifts and changes right down to the earth beneath our feet.

\item {} 
To consider the way that islands are shaped by a dynamic interplay of territories and flows.

\item {} 
To explore the inevitable entanglement of human life with non\sphinxhyphen{}human forces and processes.

\end{itemize}


\subsection{2 Island territories, ocean flows}
\label{\detokenize{content/session_00/Part_00_02:2-Island-territories,-ocean-flows}}\label{\detokenize{content/session_00/Part_00_02::doc}}

\subsubsection{2.1 Issues of responsibility}
\label{\detokenize{content/session_00/Part_00_02:2.1-Issues-of-responsibility}}
The aftermath of the 2004 Indian Ocean tsunami saw an unprecedented aid effort to assist the affected regions. In the early days after the disaster, pledges of financial assistance from overseas governments were often outstripped by the generosity of their own populaces. This was a case when ordinary people around the world saw and were moved by the tragic circumstances of others far away (Rose, 2006), and they responded with gifts of money and provisions, and even with offers of their own
skills or labour.

There has yet to be a crisis of this magnitude that has been pinned to anthropogenic climate change, though warnings of large\sphinxhyphen{}scale catastrophes from scientists and activists now abound. In contrast to natural disasters, however, climate change raises the issue of a different sort of responsibility: an obligation to others that arises not simply out of an upwelling of sympathy, but out of a feeling of being implicated in the lives of island peoples and the predicament in which they find
themselves.

Iris Marion Young (2003) has written of a kind of responsibility that comes about when we recognise that we are connected by our own actions to the suffering or injustice experienced by others who may live far away from us (see Allen, 2006). What concerns us in this section is the way in which these connections operate in a case where the actions in question transform the physical world on the global scale. To begin to grasp the issues of responsibility this raises, we must also grapple with
these transformations. As this section will also argue, the concepts of territory and flow help us make sense of how the world changes, offering an understanding of events that might otherwise seem too vast, complex and chaotic to pass into the realms of political consideration.


\subsubsection{2.2 Climate change in a globalised world}
\label{\detokenize{content/session_00/Part_00_02:2.2-Climate-change-in-a-globalised-world}}
As you will recall from Reading 1A, the people of Tuvalu are now arguing that larger and more affluent nations should take responsibility for the climatic changes threatening their country. As Paani Laupepa from the Tuvalu environment ministry put it: ‘We are on the front line … through no fault of our own. The industrialised countries caused the problem, but we are suffering the consequences’ (Lynas, 2003). Before we look more closely at this charge, and the scientific evidence that is being
compiled to support it, it is important to appreciate how the problems faced by Tuvalu form part of a much larger issue that implicates other countries or regions in different ways.


\paragraph{Activity 2}
\label{\detokenize{content/session_00/Part_00_02:Activity-2}}

\subparagraph{Question}
\label{\detokenize{content/session_00/Part_00_02:Question}}
I would like you to turn to Reading 1B by Molly Conisbee and Andrew Simms (2003) entitled ‘Environmental refugees: the case for recognition’, which you will find attached below. The aim of this piece is to gain recognition for people displaced by what the authors claim is accelerating environmental degradation. It addresses the issue in a general sense, gathering evidence from around the world.

As you are reading, note how often the terms ‘global’ or ‘international’ or some variants of these words appear.
\begin{itemize}
\item {} 
What kind of relationships do you think the authors are attempting to establish between distant places?

\item {} 
What kind of image of the world are the authors seeking to convey?

\end{itemize}

Click to view Reading 1B (3 pages, 0.08MB).

Many claims are made in Reading 1B. You may have come across similar pronouncements in news media reporting. There are quite a lot of ‘ifs’ and ‘mights’ in the reading, and you should be mindful that some of the arguments are hotly contested (as we will see later in Section 4.3), although the references to Oxford University analysts, world organisations and international panels of scientists are intended to lend them a certain authority.

At the heart of Conisbee’s and Simms’s argument (2003) is a sense that the planet is being transformed in its entirety by human activity. The term ‘global’, as it prefixes the issue of climate change, points to flows or interconnectivities that link people and places over vast distances. Like Paani Laupepa from the Tuvalu environment ministry in Reading 1A, Conisbee and Simms make a strong case for the argument that what some people do on one side of the world has serious implications for the
lives of others on the other side of the world.

Human\sphinxhyphen{}induced climate change, then, is not simply a process that takes place in a globalised world. It also helps bring the question of what we mean by ‘globalised world’ into focus, adding a powerful new dimension to the idea of an increasingly interconnected planet. The issues revolving around climate change do more than simply enfold all of us into a single, unified world. The geographical imagination suggested by Readings 1A and 1B is one that draws connections across great distances, yet
also makes important distinctions between the conditions of life of people in different parts of the world. The issue of climate change prompts us to take account of flows around the globe, but also impels us to think in new ways about the countries or territories where most people live, most of the time.


\subsubsection{2.3 Divisions that matter: thinking through territories}
\label{\detokenize{content/session_00/Part_00_02:2.3-Divisions-that-matter:-thinking-through-territories}}
Without losing our focus on the planet as a whole, it is time now to return to what Paani Laupepa from Tuvalu refers to as the ‘front line’ of climate change: those islands that are particularly vulnerable to rising sea level and associated climatic hazards (Lynas, 2003). It has often been said that low\sphinxhyphen{}lying coral islands like Tuvalu or Kiribas in the Pacific Ocean, or the Maldives in the Indian Ocean, are acting as a kind of early warning system for global climate change. Sea level is expected
to rise with even a modest increase in global temperatures, both because of the contribution of melting glacial ice to the world’s oceans and because water expands when its temperature rises.

As Conisbee and Simms (2003) remind us, the Intergovernmental Panel on Climate Change (IPCC) predicts a sea level rise over this century of somewhere between 9 and 88 cm. These figures seem at once strangely precise and wildly divergent, and it is not surprising that they provoke uncertainty and fear in the inhabitants of low\sphinxhyphen{}lying islands. Moreover, islanders and coastal dwellers in tropical regions face the prospect not only of gradually rising seas, but of an increase in the incidence and
intensity of cyclones along with the temporary surges in sea level that accompany these storms. In Tuvalu, as we can see in Figure 4, high tides can also produce flooding.

\sphinxincludegraphics[width=447\sphinxpxdimen,height=285\sphinxpxdimen]{{dd205_3_004i}.jpg}

Figure 4 Tuvalu: flooding during a very high tide

As we saw from Reading 1A, the island republic of Tuvalu has begun legal proceedings against some of the nation states it considers especially responsible for generating the hazards associated with anthropogenic climate change.


\paragraph{Activity 3}
\label{\detokenize{content/session_00/Part_00_02:Activity-3}}

\subparagraph{Question}
\label{\detokenize{content/session_00/Part_00_02:id1}}
Now turn to Reading 1C (Reuters News Service, 2002a) entitled Tiny Tuvalu sues United States over rising sea level’ and Reading 1D (Reuters News Service, 2002b) entitled Tuvalu seeks help in US global warming lawsuit’. Both of the news items in the readings emerged from the second Earth Summit, a gathering of representatives of countries from around the world which convened in Johannesburg, South Africa in 2002 to pick up on discussions about environmental issues in a worldwide context.
\begin{itemize}
\item {} 
What do these reports tell us about divisions or differences within the ‘globalised world’?

\item {} 
In the light of this information, and what you have already read about Tuvalu and its people, where do you think the loyalties or attachments of the Tuvaluans lie?

\end{itemize}

Click to view Reading 1C (2 pages, 0.05MB).

Click to view Reading 1D (2 pages, 0.05MB).

The idea of ‘tiny Tuvalu’ (Reuters News Service, 2002a) \textendash{} officially the world’s second smallest nation \textendash{} taking on much larger countries like the USA has a kind of ‘David and Goliath’ feel to it. If the threat to Tuvalu has a clear global framing, the way in which the problem is being couched and responded to also seems to rely upon, if not to reinforce, a sense of separate countries or nation states. Indeed, there is a kind of ‘us’ and ‘them’ division which appears to be taking shape around
the distinction between those who are likely to suffer the most from global climate change and those who have contributed the most to the problem. Statistics about energy consumption and contribution to carbon emissions add substance to this division.

Taken together, the four readings we have looked at suggest that the people of Tuvalu \textendash{} or at least their spokespeople \textendash{} tell a story which brings together a sense of the interconnectedness of global processes with a clear focus on the predicament of their own nation and other small island nations who are similarly at risk. It is interesting to note that Tuvalu only joined the United Nations (UN) in 1999, and did so in large part to draw attention to the islands’ vulnerability to climate change.
The expense of being a UN member was easily covered by Tuvalu selling its internet domain address extension \textendash{} which happened to be ‘.tv’ \textendash{} the equivalent of the United Kingdom’s ‘.uk’, or Russia’s ‘.ru’. In 1998, a Californian company, Idealab, agreed to pay the government of Tuvalu US\$4 million each year for the next 20 years in return for selling on the .tv address code to media companies who want to signal their role in television (such as 4kids.tv, hollywood.tv and bollywood.tv).

The Kyoto Protocol mentioned in Readings 1A, 1C and 1D was an agreement of the majority of the world’s nations to limit their carbon emissions, with the onus on the most developed or heavily industrialised economies. Though it was non\sphinxhyphen{}binding, the Kyoto Protocol nevertheless articulated a basic consensus that use of fossil fuels and other carbon\sphinxhyphen{}emitting activities were in the process of affecting global climate change. Negotiations were extremely complex, but at its simplest the Kyoto Protocol
was based around the premise that countries or nation states should not persist in activities that are known to damage or threaten other countries.

Although global climate change raises new issues, the principle of nation states not impinging in harmful ways on the territories of other nation states is an old and familiar one (Batty and Gray, 1996). At the same time, it is often tensions or disagreements between nation states that bring the nation states’ distinctiveness and defining characteristics into clearer focus. In the case of Tuvalu, which achieved independence from UK rule in 1978, the issue of climate change seems also to be
giving the people an opportunity to speak of their attachment to their islands and to voice what it is they find important about the place in which they live. As Tuvalu Finance Minister Bikenibeu Paeniu said: ‘we are not encouraging people to leave because of climate change. It’s our land. It’s where we live’ (Reuters News Service, 2002b).

Consequently in a paradoxical kind of way, the very threat to the territory of Tuvalu also seems to be helping define this territory, for the people who live there, and in the eyes of the wider world. But what exactly \sphinxstyleemphasis{is} under threat? Or to put it another way: what do we mean by ‘territory’? In Section 1, I talked about changes on a globe or map and the possibility of seeing areas of land alter their shape or even disappear completely. This scenario conveys a sense of ‘territory’ that may well
feel quite familiar: that is, territory as a particular area or parcel of land. Viewed in this way, we can think of the territory of Tuvalu as the nine coral atolls lying in the South Pacific, a combined land area of around 10 square miles (26 square kilometres), which is inhabited (in 2005) by some 11,600 people.

When we view territory as a specific area of land, the kind of outlines or borders that we can see depicted on a map take on a special importance, for they play a major role in defining the territory in question. By identifying a border on a map, we can usually tell quite quickly and easily what belongs within a particular territory (an area, country or region) and what doesn’t. In other words, we can make a distinction between what is included and what is excluded: between an inside and an
outside. But borders are not only important on maps. They also tend to play a big part in defining a territory at ‘ground level’, in the lived experience of those people who inhabit a particular piece of land. This border may be some recognisable physical feature, such as a river, mountain range or coastline. Or in other instances, especially in the case of a country or nation state, it might well be a human\sphinxhyphen{}made demarcation, such as a fence or wall.

A border, however, does more than simply \sphinxstyleemphasis{divide} a territory from the world around it. There is also an important sense in which it \sphinxstyleemphasis{connects} a territory with its surroundings, which is to say that it also functions as a zone of transition or a point of passage from one place to another (Figures 5 and 6. In this regard, passing through or crossing a border can be quite a momentous event, for it often marks a significant change in the conditions or circumstances under which people live. We can
see this with particular clarity in the case of people who are desperate to cross from one country to another, those for whom a safe transit across a border can be a life\sphinxhyphen{}changing experience. But it is not only human beings whose lives may be transformed by passing through borders. As we will see in Section 3.2, crossing a border, such as the beach that separates land from ocean, can also mark an important transition for other, non\sphinxhyphen{}human, forms of life.

\sphinxincludegraphics[width=444\sphinxpxdimen,height=367\sphinxpxdimen]{{dd205_3_005i}.jpg}

Figure 5 Border crossing at Tuvalu: carrying supplies across the beach

\sphinxincludegraphics[width=445\sphinxpxdimen,height=308\sphinxpxdimen]{{dd205_3_006i}.jpg}

Figure 6 Border crossing from the USA to Mexico: queuing for customs

Borders, then, tend to play a practical and meaningful part in defining territories. But there are ways of defining territories other than through a consideration of boundaries or outlines. And there are ways of conceiving of, or experiencing, territory other than as a simple area of land. As accounts of the threat to Tuvalu seem to suggest, there is more at risk than changing outlines or diminished land area; something more is at stake than a certain area of land distributed among a certain
number of people. Indeed, with the funds coming in from the sale of their internet domain address, the Tuvaluans could probably purchase some fairly substantial real estate elsewhere in the world.

What comes through in the voices of the people of Tuvalu is an attachment to their islands: the affirming of a whole way of life that is bound up with the place they live. And this points to a way of thinking about and experiencing territory not simply as a pocket of land, but more as a set of relationships. You might recall that in Reading 1A journalist Mark Lynas (2003) spoke of ‘a world that seemed to be unravelling’. ‘Unravelling’ suggests that a whole weave of ties or connections is being
undone, evoking a sense of territory as a kind of pattern or fabric in which many different things are bound together.

In this sense, we might still conceive of territory as a particular parcel of land along with the borders around it, but we should keep in mind that there is a lot more going on as well: an interweaving of land with all its inhabitants and their ways of life. Viewed in this more complex sense, territory is not so easy for maps to depict. While a map, a globe or a satellite photograph of a region might give us the broad outlines of territories, it is difficult for such static images to reveal
relationships between different elements and the patterns they form. But together with land areas and borders, it is this tangle of relationships that holds a territory such as the islands of Tuvalu together, giving it coherence and a character or identity of its own. In this regard, we might think of the various problems associated with changing climate as tugging at the weave of island territories, their increasing severity raising the chance that the whole might undo or come apart. In the
extreme case, the scenario that former Prime Minister Toaripi Lauti talks of in Reading 1A, the people of Tuvalu may shift elsewhere. More than a simple move to another land, this could be viewed as entailing an unravelling and re\sphinxhyphen{}weaving of relationships.

In this section we have used the concept of territory as a way to help make sense of what it is that climate change threatens; what it is that people rally together to defend against such threats. The territory we have focused on is a group of atolls in the midst of the Pacific, which can be viewed at once as discrete pockets of land bordered by ocean, and as a weave of different elements. Each of the subsequent chapters in this book will look at other examples of territory, on a range of
different scales \textendash{} some much larger than these islands, others very much smaller. And in each case, we will see that what defines a particular territory is not just its size or shape and the borders which surround it, but also the relationships among the various things or ‘ingredients’ that make it up.

But once we start to think about territories as having borders, which are points of passage as well as barriers, and once we begin to consider the different elements of which territories are comprised, it soon becomes apparent that there is little sense in looking at territories in isolation. What are also important are the things which come and go \textendash{} the movements and connections that link territories with the world around them. You will recall that we came to consider the notion of islands as
territories under threat through the issue of the interplay between connectedness \textendash{} on a global scale \textendash{} and what it felt like to be at the ‘ground level’ of climate change in a particular place. Just as it is helpful to have a sense of how territories have their own distinct identities, which can really matter to all those who live there, it is important to understand how connections or flows play a part in the lives of territories. Section 2.4 explores this notion of ‘flow’, and what it means
for our understanding of island life.


\subsubsection{2.4 Worlds in motion: the importance of flows}
\label{\detokenize{content/session_00/Part_00_02:2.4-Worlds-in-motion:-the-importance-of-flows}}
‘The sea had welled up suddenly through thousands of tiny holes in this atoll’s bedrock of coral.’ Do you recall this passage in Lynas’s (2003) account of his first days on Tuvalu in Reading 1A? For me, this gives an impression of the islands being quite literally porous, a solid ground that reveals itself, now and again, to be not so solid after all. Lynas offers this particularly striking example of the island’s openness to the world around it as evidence of a growing vulnerability that
results from global climate change. How else are islands open to the goings\sphinxhyphen{}on in the wider world? And just how novel are the openings or susceptibilities that climate change might bring to island territories?


\paragraph{Activity 4}
\label{\detokenize{content/session_00/Part_00_02:Activity-4}}

\subparagraph{Question}
\label{\detokenize{content/session_00/Part_00_02:id2}}
Now take another quick look through Readings 1A\textendash{}D attached again below for your convenience. From what you have read in these excerpts, and in the course so far, what are the different ways that the islands of Tuvalu are open to or connected with the world beyond their shores? You may have to use your imagination a little and read between the lines. As you come up with ideas, it is worth pausing for a moment and considering what difference global climate change makes to these relations or
connections. Are these new connections \textendash{} or are they connections that have been in place for a long time?

Click to view Reading 1A (2 pages, 0.08 MB).

Click to view Reading 1B (3 pages, 0.08 MB).

Click to view Reading 1C (2 pages, 0.05 MB).

Click to view Reading 1D (2 pages, 0.05 MB).

In the readings, there seems to be a number of new connections or flows that are closely related to the climate change issue \textendash{} especially the beginning of a new movement or migration of people, starting with the resettlement of some Tuvaluans in New Zealand. There is also the participation of representatives of Tuvalu in international institutions such as the UN, the International Court of Justice and the Earth Summit. In each case, climate change has been an important impetus to the Tuvaluans
establishing or extending their connections with global communities.

There are other forms of interconnection that you may have picked up on, such as air travel which is related to climate change through its contribution to carbon dioxide emissions, and internet connections which bear less of a direct relation to climate change issues though they may play a part in communicating these issues. You may also have thought about the economic goods or products that enter Tuvalu or are exported to other countries \textendash{} whether by air or sea. This serves as a reminder that
the sea is not simply an element or force that threatens the islands, but also a medium of connectivity. In a simple, intuitive way, it is the sea that separates the islands of Tuvalu from other islands or land masses, forming an obvious border or edge in a territorial sense. Nevertheless, the sea is also a way of travelling to and from the islands, and in this sense it has long played a vital role in island life, as we will see in Section 3.

Furthermore, of course, climate itself is a matter of connections. Climate change, as we have seen, implicates Tuvalu in flows of air and water that may be in the process of transformation because of anthropogenic inputs. At the same time, it is important to keep in mind that weather or climatic systems must already have been operating ‘globally’ in order for these transformations to take place, a point we will be returning to in Section 4.

Islands and other territories may have discernible boundaries, then, but a great many things pass into, out of, over or through these boundaries. Such flows implicate the lives of those in each territory with those living in other territories in many different ways. The flow of economic goods from one country or territory to another, for example, draws the people of these countries together. As John Allen (2006) argues, the everyday lives of those who live in affluent countries are entangled
with the working lives of people in relatively poorer countries through such practices as shopping for clothes and other goods that have been manufactured in sweatshop conditions.

In a similar sense, current understandings of human\sphinxhyphen{}induced climate change point to entanglements between people in distant territories. Changing flows brought about by altering the composition of the earth’s atmosphere connect distant places in a very physical way. Climate science makes the case that every single unit of non\sphinxhyphen{}renewable energy that is consumed, anywhere in the world, makes a small, cumulative addition to the planet’s overall energy budget \textendash{} therefore impacting on the global
climatic system as a whole.

Hypothetically, the energy consumed by the people of any one particular territory has an effect on every person in every territory across the planet’s surface, though in practice, the actual amount of this impact may be infinitesimal. This, as you may imagine, is a very complicated kind of entanglement indeed. It can be difficult enough to trace all the different transactions that bring a consumer of a manufactured item in one part of the world into contact with the person who produces the item
in a faraway country. Yet the lines of connection or flow that link all of us together across the planet through our respective energy use are almost unthinkably complex.

For all that the precise lines of connection between our lives and the lives of distant others may be difficult to disentangle, global issues like climate change may be helping to transform the way we experience our world \textendash{} contributing to new feelings of shared problems and common interests that span oceans and hemispheres. An understanding of how changing flows can threaten distant territories, gnawing at their boundaries and unravelling their fabric, can give a powerful emotional charge to
such a sense of connection or entanglement. However, we have to be careful that the attention given to new and far\sphinxhyphen{}reaching flows \textendash{} especially those flows that may endanger territories \textendash{} does not leave us with the impression that these territories were once free of outside influence or disturbance.

Some of the flows we have looked at are certainly disturbing, but some of them are also sustaining and generative. Indeed, it is difficult to imagine any territory maintaining itself without such flows. Similarly, your own body, although it is discernibly individual and distinct from other people’s bodies, remains utterly reliant on things passing into it, flowing through it and passing out of it. In this sense, it is more useful to conceive of flows as having an ongoing and dynamic relationship
with territories. Just as there are many different forms and compositions that territories take, so too are there many different kinds of flow. While some of these flows may help territories to form and consolidate themselves, others may exert stress and pressure upon them.


\paragraph{Summary}
\label{\detokenize{content/session_00/Part_00_02:Summary}}\begin{itemize}
\item {} 
There is growing evidence that island territories are vulnerable to changes in climate triggered by the actions of people living in other parts of the world.

\item {} 
One way of conceiving of a territory is as an area of land surrounded by a border. This border serves both to divide the territory from the world around it, and to connect it with this wider world.

\item {} 
Another way of viewing territory is as a kind of pattern or weave composed of the relationships between different elements.

\item {} 
Different kinds of flow move within and between territories, keeping them in contact and in ongoing interchange with the surrounding world.

\item {} 
Territories and flows interact dynamically; flows can help to generate territories but can also destabilise them.

\end{itemize}


\subsection{3 Settling islands}
\label{\detokenize{content/session_00/Part_00_03:3-Settling-islands}}\label{\detokenize{content/session_00/Part_00_03::doc}}

\subsubsection{3.1 Voyages of discovery and settlement}
\label{\detokenize{content/session_00/Part_00_03:3.1-Voyages-of-discovery-and-settlement}}
In Section 2, we saw that there are momentous new and recently transformed flows that are impacting on island territories. Some flows have important precedents, and others may not be quite as novel as they first appear. In this section, we look more closely at some of the flows that have helped make, remake and sometimes unmake islands.

This takes us away from the flows that have captured recent attention, such as movement of goods or human\sphinxhyphen{}induced changes in climate, drawing us into the longer\sphinxhyphen{}term process of the formation of island territories. Beginning with the journeys that have taken human explorers and colonists to oceanic islands, we move on to other acts of settlement that are no less wondrous and impressive.

The people of Tuvalu, as we have seen from Readings A\textendash{}D, are contemplating leaving their islands and shifting permanently to higher and drier ground elsewhere in the Pacific. Consequently, some islanders have already left for New Zealand. Migration, at least in this context, is a kind of flow which occurs as a response to a territory felt to be under threat or pressure. If some of the predictions presented by Conisbee and Simms (2003) in Reading 1B turn out to be accurate, the flow of migrants
triggered by climate change and other forms of environmental degradation will dramatically increase over coming decades. Even without additional movements propelled by environmental causes, current rates of migration are already often referred to as ‘floods’ by people in receiving countries. Nevertheless, while there may be many new pathways and intensities of movement in the contemporary world, migration is far from being a novel form of flow.


\paragraph{Activity 5}
\label{\detokenize{content/session_00/Part_00_03:Activity-5}}

\subparagraph{Question}
\label{\detokenize{content/session_00/Part_00_03:Question}}
While you have been reading the story of the threatened existence of the Tuvaluans, have you stopped to wonder how they came to be living on these islands in the first place?
\begin{enumerate}
\sphinxsetlistlabels{\arabic}{enumi}{enumii}{}{.}%
\item {} 
How did the Tuvaluans come to be hundreds of kilometres from any other land, out in the wide open waters of the western Pacific?

\item {} 
Did the Tuvaluans’ ancestors once discover these islands? And, if so, where did they come from?

\end{enumerate}

Europeans often talk about having ‘discovered’ many oceanic islands during an era of maritime exploration between the sixteenth and nineteenth centuries, when voyagers like Ferdinand Magellan and James Cook and their crews sailed through the Pacific. However, this is rather misleading for, as anthropologist Greg Dening (1992) reminds us, by this time the Pacific had already been thoroughly explored. As he tells the story:


\begin{quote}

There are more than 25,000 islands in the Pacific. Yet any one of them can be lost in an immense ocean that covers a third of the globe. Remarkably, in the central Pacific where a canoe or a ship could sail for months or for 5,000 miles and never make a landfall, every mountaintop that had pushed from the ocean bed, every coral reef that had grown above the ocean surface had been discovered before the European strangers had had the courage or the knowledge or the technology to discover the
sea. (Dening, 1992, p. 307)
\end{quote}

Evidence suggests that the people who first populated the Pacific, whom anthropologists refer to as ‘Austronesians’, departed from eastern\sphinxhyphen{}most Asia and the islands off Southeast Asia. As geographer Patrick Nunn points out, leaving the mainland, or islands that are densely\sphinxhyphen{}packed and often visible one from another, and heading out into the open ocean where islands are hundreds or thousands of kilometres apart, would have presented an enormous challenge (Nunn, 2003). Moreover, as he reminds us,
such voyages would have commenced in the context of a very different geographical imagination than the one many of us share today. As Nunn puts it: ‘It is difficult today to imagine ourselves without our knowledge of the world. We know the geography of the earth’s surface, we have only to flick open an atlas to know instantly the bounds of the Pacific Basin, but the first islanders did not’ (Nunn, 2003, p. 222).

It is now believed that Austronesian peoples were the first in human history to master long\sphinxhyphen{}distance ocean sailing, and that they began to colonise the islands of the western Pacific some 3000\textendash{}4000 years ago. Those who later continued to journey eastwards into the Pacific settled the islands of Polynesia. Known today as ‘Polynesians’, these people continued their way eastwards across the Pacific as far as South America and southwards as far as New Zealand (Aotearoa). Other Austronesians headed
westwards across the Indian Ocean, eventually settling in Madagascar, off the coast of Africa (see Figure 7).

\sphinxincludegraphics[width=579\sphinxpxdimen,height=313\sphinxpxdimen]{{dd205_3_007i}.jpg}

Figure 7 The Austronesian diaspora with estimated dates of colonisation (Source: based on Pyne, 1997, p. 419 and Fischer, 2002, Ch. 1)

Reference:

For a long time, Western anthropologists and historians toyed with the idea that most islands were discovered and settled accidentally, by sailors swept away from familiar waters. However, the fact that enough men and women arrived on newly discovered islands to create viable populations, and that there is evidence that they usually arrived with a whole range of plants and animals which they relied upon for food and other needs, suggests a much more organised pattern of settlement (Hau’ofa,
1993, p. 9).

Flying over the Pacific, I have looked out of the aeroplane window, trying to spot small islands. It seems like you can fly for hours without seeing even the tiniest speck of land, which makes you wonder how those early navigators ever found their islands or, having left, ever found them again. This is especially intriguing in the case of Tuvalu and other atolls which are, in the most part, no more than a few metres above sea level.

Those who have studied traditional Pacific navigation give accounts of seafarers gradually building up, over thousands of years, knowledge of swell patterns, wave refraction, currents, prevailing winds and the position of stars. Seafarers were also familiar with more ephemeral signs \textendash{} phosphorescence, the colour or shape of clouds, the presence of certain fish or birds. By reading such signs, traditional navigators could precisely locate a speck of land in a vast ocean \textendash{} a practice known in
nautical terms as ‘dead reckoning’. Moreover, they could still sail home in this way in cases where their vessels were storm\sphinxhyphen{}blown hundreds of kilometres off course (Lewis, 1994).

Marshall Islands stick ‘charts’, such as the one in Figure 8, are used for teaching about wave refraction around islands. Unlike most modern Western charts or maps that attempt to give a one\sphinxhyphen{}to\sphinxhyphen{}one correspondence with the area they represent, the stick chart is not intended to be in proportion to actual oceans and islands, and it does not necessarily refer to any specific area. Instead, it depicts the processes or dynamics by which swells hitting an island are refracted back into the ocean. It
is for learning purposes only and is not taken to sea. It is said that traditional Marshallese seafarers could lie in the bottom of their canoes and navigate using the feel of waves and the current on the hull. This suggests that they relied less on visual recognition and cues than do most modern Western mariners.

\sphinxincludegraphics[width=286\sphinxpxdimen,height=233\sphinxpxdimen]{{dd205_3_008i}.jpg}

Figure 8 Stick chart from the Marshall Islands

Amid all the contemporary talk of accelerating long\sphinxhyphen{}distance migration, tourism and other flows of people around the world, it is easy to overlook how far and how frequently people travelled hundreds or even thousands of years ago. In the late eighteenth century, Captain Cook noted that Polynesian ocean\sphinxhyphen{}going canoes could sail far faster than his own ships, and he judged that they could ‘with ease sail 40 Leagues {[}120 miles{]} a day or more’ (cited in Lewis, 1994, p. 70). More recent evidence
supports Cook’s estimates, pointing not only to long\sphinxhyphen{}distance journeys around the Indian Ocean and across the Pacific, but also to very frequent trips between neighbouring island groups.

Pacific scholar Epeli Hau’ofa speaks of Pacific islanders, prior to European contact, engaging in a constant movement of ideas, goods and people which linked distinct island groups or territories. As he explains: ‘Fiji, Samoa, Tonga, Niue, Rotuma, Tokelau, Tuvalu, Fatuna and Uvea formed a large exchange community in which wealth and people with their skills and arts circulated endlessly’ (Hau’ofa, 1993, p. 9). After much of the Pacific was colonised by Europeans, colonial administrators tried to
restrict inter\sphinxhyphen{}island voyaging in an attempt to pin down and firm up the boundaries of various island territories. Hau’ofa tells of an earlier time: ‘the days when boundaries were not imaginary lines in the ocean, but rather points of entry that were constantly negotiated’ (Hau’ofa, 1993, p. 9).

Hau’ofa seems to be saying that islands, though they may be bounded in some respects, were certainly not closed or isolated. These were territories that were permeated by flows. Therefore, centuries and, in some cases, millennia before Europeans made their way across the world’s oceans, Pacific navigators were already reworking sea and islands into a space in which human beings ‘flowed’. Although it is unlikely that they would have been able to imagine the world to be a single place \textendash{} in the way
it is now possible to think of the globe in its entirety \textendash{} Austronesian and Polynesian voyagers succeeded in forging connections that spanned more than half of the planet’s surface. In fact, with hindsight, it has been argued that these seafaring peoples made greater leaps towards globalising the world than any others, before or since (Gould, 1992, p. 109).

Looking at this long history of oceanic journeying helps to give a sense of the way human beings have been generating new flows \textendash{} sometimes over very long distances \textendash{} for thousands of years. These flows have been vital in establishing new territories by opening up lands for settlement. Nonetheless, it is important to see that the settling of new islands has not been achieved by humans alone. As was suggested above, Polynesian settlers travelled with a ‘portmanteau’ of useful animals and plants.
Along with seedlings of the plants they needed for food and clothing, island colonists across much of the Pacific also introduced their traditional ‘feasting’ animals \textendash{} pigs, dogs and chickens \textendash{} to their new homes (Dening, 1992, pp. 307\textendash{}8). In the case of Tuvalu, such staple foodstuffs as breadfruit, taro and banana would probably have been brought to the islands on board the canoes of inter\sphinxhyphen{}island voyagers.

It is worth considering more closely the dynamic relationship between territories and flows in the case of oceanic islands. We have seen in Section 2.3 how the concept of territory helps us to conceive of islands as a weave of many different strands. Like other territories, islands are inconceivable without the input and throughput of flows and, as with the notion of territory, one of the advantages of thinking through the concept of flow is that it can be inclusive of both human and non\sphinxhyphen{}human
elements. Thus, when we consider the ways in which territory and flow are interrelated, it is possible to address the human and the non\sphinxhyphen{}human together, and to recognise that they often share similar dynamics.

Considering the interaction of territory and flow encourages a view of islands as having been made, rather than simply discovered or awaiting discovery. This making is ongoing: islands remain open to the possibility of being unmade or remade. However bounteous and balmy tropical islands may sometimes appear, we should not forget that making islands is difficult and often dangerous work. Every new arrival to an island \textendash{} either human or non\sphinxhyphen{}human \textendash{} has to find some way of weaving itself into the
existing fabric of island life if it is to make itself at home. On smaller islands especially, newcomers may struggle to find enough of the things they need to sustain them, while the existing pattern of island life may be deeply disturbed by even a few impetuous new arrivals. In many cases, not only have new groups of human settlers caused much damage to the islands on which they have settled, but so too have the rats, pigs, cats and other predatory species that have accompanied such humans on
their oceanic voyaging (Quammen, 1996).

Yet is it enough to think of human beings, working together with their companion species, as the producers of viable island territories? Human colonists did not settle barren rocks or bare coral in the middle of the ocean. They, and the useful species they brought with them, could never have settled themselves were the islands they found not already a rich weave of living and non\sphinxhyphen{}living things. If the story of how the earliest oceanic voyagers established new flows between distant lands is an
awe\sphinxhyphen{}inspiring one, no less epic are the achievements of all the other life forms which had already made themselves at home on even the most isolated oceanic islands.


\subsubsection{3.2 Migrations of life}
\label{\detokenize{content/session_00/Part_00_03:3.2-Migrations-of-life}}
As biologist and pioneer environmentalist Rachel Carson once wrote: ‘the stocking of the islands has been accomplished by the strangest migration in earth’s history \textendash{} a migration that began long before man appeared on earth and is still continuing’ (Carson, 1953, p. 66). Austronesian voyagers may have been the first people to venture far into open water, but many other species, as Carson suggests, have also found ways of negotiating passages across the ocean. Arriving at pockets of land
thousands of kilometres out in the Pacific, the first human voyagers encountered plants, insects, crustaceans, birds and sometimes even reptiles and mammals.


\paragraph{Activity 6}
\label{\detokenize{content/session_00/Part_00_03:Activity-6}}

\subparagraph{Question}
\label{\detokenize{content/session_00/Part_00_03:id1}}
In the same way that Activity 5 asked you to ponder how the people of Tuvalu came to inhabit their islands, now give some thought to how other forms of life may have reached oceanic islands before they were able to hitch lifts on human vessels. It may take a little imagination, and an ability to take low odds and sheer fluke into account, but see if you can come up with some ideas.

This is indeed a challenging problem, and biologists, including Charles Darwin, have spent a great deal of time trying to answer it. Back in the mid nineteenth century, when Darwin was still working on his theory of evolution, biologists, or ‘natural historians’ as they were usually called, often relied on theories of land bridges to explain how different organisms found their way to islands. At this time it was already widely accepted that sea level had varied considerably over long periods of
time, so that present\sphinxhyphen{}day islands may not always have been encircled by sea. As Darwin wrote in \sphinxstyleemphasis{The Origin of Species}, ‘authors have thus hypothetically bridged over every ocean, and have united every island to some mainland’ (Darwin, 1996, pp. 288\textendash{}9, first published 1859). While agreeing that these submerged connections might explain the populating of some islands, particularly those closer to the mainland, Darwin was dubious about their extension to oceanic islands \textendash{} in part, because he was
well aware that many of these islands were volcanic in origin, rather than being detached outcrops of once larger continents.

This left Darwin speculating over, and experimenting with, the ways that various forms of life might have made it across oceans. He proposed that seabirds must have played a major part in the dissemination of plant life by unintentionally dispersing seeds. Following an intuition along this line, Darwin once plucked a ball of mud from the plumage of a seabird, and extracted from it enough seeds to grow 82 separate plants from five different species (Carson, 1953). He performed similar experiments
with seeds that had passed through the digestive system of birds, and also gathered evidence that nuts, fruit and other seed\sphinxhyphen{}bearing propagules could endure lengthy immersion in sea water and still germinate successfully (Darwin, 1996, pp. 290\textendash{}3, first published 1859). Later researchers have gone on to identify lightweight seeds equipped with feather\sphinxhyphen{} or parachute\sphinxhyphen{}like appendages that enable them to ascend high into the sky and to be wind blown across vast distances, as Rachael Carson (1953)
recounts.

While sea\sphinxhyphen{}going birds and ocean and air currents could explain the dispersal of plant life, many members of the animal kingdom have posed thornier problems. The presence of land\sphinxhyphen{}based and frequently flightless birds on oceanic islands \textendash{} such as the ill\sphinxhyphen{}fated dodo of Mauritius \textendash{} was especially baffling. That is, until Darwin’s theory of long\sphinxhyphen{}term evolutionary change offered a way of explaining how birds that were once capable of long\sphinxhyphen{}distance flight had subsequently adapted physically to a more
terrestrial life. While some insects have likewise arrived with their own wing power, others have evolved ways of hitching lifts with seabirds or drifting on air currents, such as the numerous species of spider known to float or ‘balloon’ across the sea on their own silken filaments (see Figure 9) (Winchester, 2004; Carson, 1953).

\sphinxincludegraphics[width=287\sphinxpxdimen,height=369\sphinxpxdimen]{{dd205_3_009i}.jpg}

Figure 9 Nephila maculata: a ballooning spider

In the case of many land\sphinxhyphen{}based life forms, which could not conceivably have relied on wings, fins or floating ability, it was Darwin’s contemporary, Alfred Wallace, who offered evidence from the field of some intriguing modes of maritime voyaging. Among the islands between the Pacific and Indian Oceans, Wallace observed large clumps of drifting vegetation, and speculated that all sorts of organisms \textendash{} including mammals and reptiles \textendash{} could raft across hundreds of kilometres of open sea in this
way, eventually colonising new islands. However, he also recognised that this was an incredibly chancy affair, especially considering that many species rely on sexual reproduction, and that it would require a pair of organisms to complete the journey (Quammen, 1996, p. 145). Subsequent researchers have lent support to Wallace’s speculations. It is now widely acknowledged that not only uprooted trees and matted vegetation but also volcanic rock, such as the eminently buoyant pumice, operate as a
‘sea going transport system’ for the propagation of island life (Barnes, 2002, p. 808).


\paragraph{Defining colonisation}
\label{\detokenize{content/session_00/Part_00_03:Defining-colonisation}}
Colonisation refers to the process by which organisms become established in an area where they were not previously found. It is a normal and vital part of the changing distribution of life across the planet. In this sense of the word, there have been many times when human beings, like any other species, have established themselves in areas where there were previously no people. In cases where human colonisation involves the establishment of a new group of people at the expense of people
previously inhabiting an area, colonisation takes on other, political meanings, not usually associated with colonisation as a biological process (Barnett, 2006).

Many of the ocean\sphinxhyphen{}crossing or island\sphinxhyphen{}hopping journeys that have brought new species to islands must have been so rare that their contribution to island life only makes sense if great reaches of geological time are taken into account. Being so discontinuous and fitful, it may be stretching the use of the concept to refer to all such forms of mobility as flows’ What is remarkable about these dispersals of non\sphinxhyphen{}human life is their continuity with some of the different modes or means of human
migration. Polynesian seafarers, you may recall, found their way to new islands by taking advantage of winds and ocean currents, as well as taking pointers from seabirds and ocean\sphinxhyphen{}going life forms. These are much the same physical or elemental forces that other species have relied upon to reach new islands, albeit in a more haphazard fashion. Consequently, we might also refer to the prevailing winds and ocean currents, as well as the more regular movements of living things, as forms of ‘flow’ \textendash{}
flows which can be tapped into by opportunistic travellers, both human and non\sphinxhyphen{}human.

Islands have been shaped and woven into territories through a combination of flows of very different kinds. Without the many other organisms that have made it over the ocean, crossed the beach and performed the difficult task of weaving themselves into a workable web of life, new human arrivals could not have made a home of these islands.


\paragraph{Summary}
\label{\detokenize{content/session_00/Part_00_03:Summary}}\begin{itemize}
\item {} 
Early seafarers developed the skills to take advantage of flows of wind and sea in order to colonise oceanic islands, as well as creating their own flows between islands.

\item {} 
Other forms of life have also taken advantage of flows of wind and sea in order to cross the sea and colonise oceanic islands.

\item {} 
Every new arrival \textendash{} whether human or non\sphinxhyphen{}human \textendash{} that crosses the ocean must weave itself into the existing territory if it is to become established on an island.

\item {} 
Human and non\sphinxhyphen{}human beings often share similar dynamics in the utilisation of flows to find and shape territories.

\end{itemize}


\subsection{4 Volatile worlds}
\label{\detokenize{content/session_00/Part_00_04:4-Volatile-worlds}}\label{\detokenize{content/session_00/Part_00_04::doc}}

\subsubsection{4.1 When climate changes}
\label{\detokenize{content/session_00/Part_00_04:4.1-When-climate-changes}}
We have seen that human\sphinxhyphen{}induced climate change poses a challenge for people who live on islands. Such changing patterns and extremes of climate also put pressure on the other living things that are part of the make\sphinxhyphen{}up of island territories. However, long before human beings became aware that they could transform the flows that constitute climate, they and other species were already taking advantage of these same flows to help create the very territories that are now under threat. But have these
flows themselves changed over time, even without human input? As mentioned in Section 3.2, it has been known at least since Darwin’s day that sea levels have changed dramatically over time. This raises some interesting questions about contemporary changes in climate, and how we respond to their effects. In this section, we explore some of the ways in which the earth itself shifts, including the flows we call ‘weather’ or ‘climate’, and the very ground beneath our feet. This takes us into the
realm of momentous changes that occur even without human impact or influence, which in turn raises some thorny questions for the issue of responsibility.

As you may recall from Section 3.1, anthropologists suggest that the early settlers of the Pacific probably departed from eastern Asia, perhaps via the islands off Southeast Asia. From here, they set out into the world’s largest reach of open water, with little way of knowing what they might find.


\paragraph{Activity 7}
\label{\detokenize{content/session_00/Part_00_04:Activity-7}}

\subparagraph{Question}
\label{\detokenize{content/session_00/Part_00_04:Question}}
In Activity 5, the question was posed of how the Tuvaluans came to be on their islands in the first place. You may now have some idea of how they got there, but we have not really considered the question of why. Why do you think the distant ancestors of the people who eventually arrived in Tuvalu (most likely after hundreds of years of island\sphinxhyphen{}hopping) left their original homelands?

Hau’ofa (1993) and other scholars familiar with the traditions of Pacific seafaring present a good case that sheer adventurousness has played a large part in Pacific voyaging. Yet this claim refers to people who have had thousands of years of experience in deep\sphinxhyphen{}water sailing and navigation. It may not be so useful in accounting for those first forays into the ocean, away from the sight of land.

Perhaps some pressure or stress helped to push people seawards \textendash{} possibly a shortage of land or other resources. Something along these lines has been suggested by Nunn (2003), and his explanation turns out to have rather a surprising relevance to contemporary islanders threatened by the effects of climate change. Nunn proposes that it might have been the stress brought on by changing climate that first propelled people out into the ‘blue water’ of the Pacific.

It is, of course, very difficult to piece together the motivations of people who lived thousands of years ago, and so speculation is called for. Nevertheless, based on a combination of evidence from archaeology and climatology, Nunn (2003) propounds that rising sea levels caused by a cycle of long\sphinxhyphen{}term climatic warming may have been a major push factor.

In the aftermath of the last ice age (about 10,000 years ago), the receding and melting of glaciers would have led to rising sea levels all around the world, with serious repercussions for people who had settled into agricultural life on coastal lowlands. One likely area of displacement, a region where it is known that there were early farmers, was the coasts of East Asia, especially on the rich alluvial plains around the mouths of large rivers like the Huanghe and Yangtze in present\sphinxhyphen{}day China.
Some of these people, Nunn suggests, may have headed out into the ocean: ‘In this scenario the first true Pacific Islanders were “environmental refugees” rather than the bold, curious adventurers they are sometimes portrayed as having been’ (Nunn, 2003, p. 220).

However, it is unlikely that everybody left the coasts and set out into the oceans, so we might amend Nunn’s view and say that they may have been bold and adventurous as well as being under pressure! If we keep in mind, as Nunn encourages us to do, that those setting out into the Pacific had no way of knowing what the world into which they were heading was like, then it is conceivable that those people who ventured out into the open ocean may have been hoping to find places where the sea was not
rising (Nunn, 2003, p. 22).

Another way to express this would be to say that as a certain territory came under pressure, or began to destabilise or come undone, some people began an outward flow that would eventually lead them into making new territories. As Lynas (2003) observed in Reading 1A, under current conditions of climatic change the world ‘seemed to be unravelling’. One implication of Nunn’s (2003) view is that the world may have also felt like it was ‘unravelling’ in the past, perhaps many times before. Nunn’s
argument makes a strong claim for the ongoing impact of climatic change on the peoples of the Pacific. For the people of Tuvalu, and many other low\sphinxhyphen{}lying islands, climate change indeed appears to have been decisive in the past. As Nunn explains, it was only after a colder period 2000 or 3000 years ago, with resultant falls in sea level, that the land that became Tuvalu surfaced from the ocean, enabling the accumulation of sand and gravel, and coral growth that helped make the islands.

Subsequently, from around AD 750 to 1250, the Pacific experienced a phase of gradual warming known to climate change scientists as the ‘Medieval warm period’ (Nunn, 2003, p. 223). As temperatures rose, so too did sea levels. Rising sea levels would have brought salt into the fresh water beneath the ground of many islands, as it is doing at present, while declining rainfall would have increased aridity. Although such conditions would have impacted harshly on island life, there may have been some
compensation as clear skies and decreased storminess seem to have encouraged long\sphinxhyphen{}distance voyaging \textendash{} increasing inter\sphinxhyphen{}island contact and leading to the discovery and settlement of new islands.

Around AD 1300, this warming came to a close and a period of rapid cooling followed. In Europe, this was known as the ‘Little Ice Age’. Global cooling meant more water locked up in ice as well as a general lowering of sea water temperatures worldwide, with the result that sea level may have fallen by as much as 1.1 m between the thirteenth and sixteenth centuries. There are paintings and prints from this period of English history showing people ice\sphinxhyphen{}skating or walking on a frozen River Thames.
While this spell of coldness may have caused considerable hardship in Europe and other temperate regions, global cooling had rather different implications for the island peoples of the Pacific. Nunn (2003) describes the likely effects of what he calls the ‘AD 1300 Event’:


\begin{quote}

Almost all Pacific Islanders at the time lived along island coasts and, although they may have had inland food gardens, they would also have depended on crops (including coconuts) growing on coastal lowland areas. As sea level fell, so water tables fell, and many such crops would have grown and yielded less well. More importantly, these people would have been accustomed to acquiring food from nearshore coral reefs but when the sea level fell the most productive parts of these reefs would have
been exposed above sea level and would have died. Likewise these people would have routinely exploited lagoonal resources for sustenance or trade, but when the sea level fell and exposed the surfaces of the nearshore reefs, this would have inhibited lagoonal water circulation resulting in turbidity and sluggishness, which in turn would have caused a deterioration in the health of lagoon ecosystems and a decline in their organic productivity. The cooling and the increased storminess during the
AD 1300 Event would have exacerbated many of the effects described above, largely through increasing stress on various food\sphinxhyphen{}producing ecosystems. It is thought that within 100 years of the AD 1300 Event, the food resources readily available to coastal dwellers in the reef\sphinxhyphen{}fringed Pacific Islands fell by around 80 per cent. (Nunn, 2003, pp. 223\textendash{}4)
\end{quote}

It is difficult to piece together all the factors involved, but such changes are likely to have brought great stress and loss of life: what we might describe as a serious undoing and remaking of many Pacific island territories. Moreover, in the case of very low\sphinxhyphen{}lying islands like Tuvalu there was no option, as there was on larger islands, of moving inland to exploit new resources. One redeeming factor is that falling sea levels would have brought new islands and atolls to the surface, some of
which were then settled (Nunn, 2003, p. 224). Such islands, however, would have been poor in resources, having not had time to develop a rich web of living organisms.

Nunn’s (2003) research reminds us that climate fluctuates and varies, even without human impact. More than this, the evidence he presents makes the point that climate changes not just over millions or thousands of years, but sometimes over timescales short enough to be registered in human lifetimes or in memories passed between generations. While changing flows of air and water might currently threaten island territories like Tuvalu, we need also to consider that earlier changes have not only
disrupted island life before, but may have played a pivotal role in initiating the exploration of oceans and settlement of islands. Changing climate contributes not only to the unmaking of islands, but also to their making.

We have seen that there were very different, but nonetheless parallel, experiences of cooling in the AD 1300 Event in northern temperate countries like Britain and on the tropical islands of the Pacific. It is worth considering what the implications of this might be for the way we imagine our world. In both places, at the same time, climate change exerted considerable stress. Yet, from a human perspective, these far apart territories did not share a common world. It was only towards the end of
the cold phase that the first European explorers ventured into the Pacific Ocean, and it was not until several centuries later that contact between Europeans and Pacific islanders became ongoing and sustained. Therefore, in the imaginary geographies of the era \textendash{} of both Europeans and islanders \textendash{} it was not yet possible to experience climate change as a fully global phenomenon.

Nonetheless, as we can see through the lens of a contemporary world view, climate change in the AD 1300 Event and all other major fluctuations in climate are fully global phenomena. Climate in temperate Europe and climate in the tropical Pacific was as connected and mutually implicated then as it is now, as it has been for the estimated 4.5\sphinxhyphen{}billion\sphinxhyphen{}year existence of our planet. Thus, knowledge of human\sphinxhyphen{}induced climate change may be contributing to a new experience of entanglement between faraway
people or places, but in another sense \textendash{} a purely physical sense \textendash{} our planet has always already been fully globalised.

An understanding of the globalised climatic flows that impact on oceanic islands adds further weight to the idea that the making of territories is a much more than human process and, indeed, a much more than biological or organic process. This awareness of forces and energies that are literally larger than life becomes even more pronounced when we turn to the physical processes that bring islands into existence.


\subsubsection{4.2 Shifting ground}
\label{\detokenize{content/session_00/Part_00_04:4.2-Shifting-ground}}
In Section 3 and in Section 4 so far, we have begun with the questions of how and why humans found their way to oceanic islands, and how other living things have come to make themselves at home on these same islands. The question we have yet to consider, the one that in a way underpins these other questions, is how there came to be isolated tracts of land in the middle of a vast ocean in the first place. To answer this, we need to turn to the insights of the earth sciences.

There are hints that have surfaced in previous sections of the unit about the formation of islands. As you may recall from Section 4.1, falling sea levels facilitated the accumulation of sand and gravel, and the coral growth that helped form the islands that are today’s Tuvalu. However, in order for these processes to occur, there must already be a significant protrusion from the seabed, and it is the formation of such irregularities that directs our attention to some of the most powerful forces
that have shaped, and continue to shape, our planet.


\paragraph{Defining earth science}
\label{\detokenize{content/session_00/Part_00_04:Defining-earth-science}}
Earth science (also known as ‘geoscience’) is an all\sphinxhyphen{}embracing term for the sciences related to the study of the origin, structure and physical phenomena of the planet earth. It includes the study of rocks, oceans and fresh water, ice and the atmosphere, as well as the dynamics that connect these parts of the planet together.

As most earth scientists now agree, it is the movement of the vast rigid plates that make up the earth’s crust \textendash{} the process of plate tectonics \textendash{} that is behind the formation of the major peaks and ranges that rise up from the surface of the planet. The location of these plates, and their direction of movement, can be seen in Figure 10.

\sphinxincludegraphics[width=443\sphinxpxdimen,height=316\sphinxpxdimen]{{dd205_3_010i}.jpg}

Figure 10 The location of tectonic plates, showing direction of plate movement and sites of volcanic activity (Source: Colling et al., 1997, p. 113)

Reference:

The molten rock (or magma) that creates these tectonic plates wells up from beneath the earth’s crust along submarine ridges on the seabed. In turn, this creation of new crust pushes the existing plates sideways, where they collide with other plates. When plates converge in this way, one of them will be forced downwards, deep below the earth surface, in a process of plate destruction that balances out plate creation. The immense force of one plate being driven under the other melts the crust
into magma. When this occurs on continental land masses, this magma tends to erupt into volcanic mountain ranges; when it occurs under the sea, the resulting eruptions tend to give rise to an arc of volcanoes which forms the basis of oceanic islands such as those found in the western Pacific (Colling et al., 1997, pp. 114\textendash{}15). These processes are depicted below.

The action of plate tectonics, showing the formation of volcanic island arcs at the meeting point of plates In the warmer reaches of the oceans, some of the work that brings the peak or ‘cone’ created by volcanic activity to the surface can be carried out by life itself. Coral, attracted to a submerged volcano, lives out its life just beneath sea level. The skeletons of coral polyps build up into solid structures of limestone over many generations, which may then be pushed above the waves by
further tectonic activity to form an island or atoll.

Once a cone rises above the sea’s surface, it begins to attract wandering life forms, as we saw in Section 3.2. Yet the same processes that produce the ground on which organisms can gain a foothold can also extinguish this life, reverting a verdant island to barrenness with a later coating of ash and lava, or annihilating the island altogether in a violent outburst or subsidence (Carson, 1953; Winchester, 2004).

The Indian Ocean tsunami of December 2004 was a reminder of the immense force of the ongoing process of plate tectonics. The waves were generated by an earthquake just north of the island of Sumatra, at a point along the juncture where the Indian Plate is being driven under at a rate of around 2.5 cm or so a year as a result of its convergence with the China Plate (often included as part of the larger Eurasian Plate). While a string of active volcanoes in the region usually provides a release
for the energy generated by this ongoing collision, geologists believe that a sticking point in the convergence zone led to a gradual build\sphinxhyphen{}up of pressure. This in turn caused a 965\sphinxhyphen{}km\sphinxhyphen{}long section of plate to subside suddenly, displacing vast amounts of water and producing the massive swells that swept across the Indian Ocean with such devastating results.

Unlike the flows and fluctuations in global climate, which we now understand to be at least partially influenced by human activity, the geological processes that give rise to islands and other major protrusions on the earth’s surface remain largely impervious to all the exertions of our species. Along with upwelling of magma, the shifting and shuddering of tectonic plates are evidence that even the ground beneath our feet is in flux. Furthermore, just as human migrations and excursions, the
mobility of other living things, and the currents of air and water each constitute different kinds of flow, we might also conceive of these ongoing movements of the earth’s crust as a particular kind of flow. Sometimes this flow is gradual, so slow as to be imperceptible, but at other times, as I intimated in Section 1, it manifests itself cataclysmically, in ways that can render our maps obsolete in minutes or seconds.

It may be that for most of us it is only at these occasional or exceptional moments of upheaval that the flow of the ground on which we construct our territories really makes itself felt. However, no less than the flows and circulations that comprise the earth’s climate, the flux of plate tectonics is a fully global process in which different parts of the earth’s crust work together on the planetary scale. An understanding of this geological volatility of the earth brings another dimension to
the sense of territories as constantly in the making, which we have been exploring in various ways throughout this course. In a particularly powerful way, this volatility points to the contingency of any territory, and to the possibility for unmaking that is inseparable from the potential of making and remaking. There may be no better illustration of these contingencies than oceanic islands, which are forged from some of our planet’s most convulsive processes, and brought to life by some of the
most risky and opportunistic migrations ever undertaken.

What does this mean for the plight of islanders like the people of Tuvalu who find themselves in the front line of potentially momentous transformations of global flow? In the following section, we return to the question of human\sphinxhyphen{}induced climate change in relation to some of the other active processes and forces we have been examining.


\subsubsection{4.3 Dilemmas of climate change}
\label{\detokenize{content/session_00/Part_00_04:4.3-Dilemmas-of-climate-change}}
In Section 4.1, we looked at claims that climatic change thousands of years ago triggered the movement of people into the ocean, eventually leading to the settling of islands like Tuvalu. We have also seen that these islands only rose out of the ocean because of dynamic geological processes coupled with dramatic changes in climate and sea level.


\paragraph{Activity 8}
\label{\detokenize{content/session_00/Part_00_04:Activity-8}}

\subparagraph{Question}
\label{\detokenize{content/session_00/Part_00_04:id1}}
Take a moment to consider how an understanding of the formation of oceanic islands like Tuvalu makes you feel about the current predicament of the Tuvaluans. Has it changed the way you think about human\sphinxhyphen{}induced global climate change and the responses or responsibilities this might entail?

The people of Tuvalu claim that carbon emissions are changing global climate. In particular, they point to the unwillingness of nation states like the USA and Australia to support the Kyoto Protocol, which seeks to reduce these emissions. The idea that the earth’s natural greenhouse effect is being enhanced by a build\sphinxhyphen{}up in carbon and other greenhouse gases because of non\sphinxhyphen{}renewable energy use is fully supported by many environmental campaigners around the world, such as Molly Conisbee and Andrew
Simms, the authors of \sphinxstyleemphasis{Environmental Refugees} (2003; see Reading 1B). This notion is also supported by the huge international consortium of climate scientists that comprises the IPCC. Similarly, for Nunn (2003), and many other geographers who have made a study of oceanic islands, the inherent changeability and precariousness of island life is reason not to discount human\sphinxhyphen{}induced climate change, but to take it very seriously indeed.

Nonetheless, perhaps unsurprisingly, there are others who see this differently. Some commentators have used the idea of the constant motion of the earth’s crust to counter the argument that greenhouse gas emissions are having a significant impact on sea levels. In this regard, US climate negotiator Harlan Watson has noted the particular instability of the Pacific: ‘The South Pacific is very volcanically unstable on the sea floor … so you have some natural subsidence occurring anyway. Islands are
appearing and disappearing all the time’ (cited in Kriner, 2002, p. 2). Other voices in the climate change debate have argued that because global climate fluctuates constantly, current warming is nothing out of the ordinary. Danish statistician Bjorn Lomborg, for example, has suggested that as the planet is still coming out of the Little Ice Age, a gradual warming and attendant rise of sea level is to be expected (Lomborg, 2001, p. 263). Similarly, Bill Mitchell of the Australian Tidal Facility
claims that there is little evidence in the Pacific of sea level changing more rapidly than that which would be expected from gradual natural warming (Field, 2002).

It would be an oversimplification, however, to suggest that the climate change debate neatly divides itself into two opposing factions. Researchers and campaigners of all persuasions have taken the possibility of human\sphinxhyphen{}induced changes into account as well as the many physical variations that are not reducible to human activity, and this leaves room for very different weightings to be applied to the many variables involved.

Thinking through territories and flows does not offer any easy answers to, or any direct route out of, the dilemmas posed by the prospect of a human contribution to global environmental change. Yet it can offer a versatile way to approach any such issue that brings together human and non\sphinxhyphen{}human forces. Furthermore, with the kind of issues that present themselves in the contemporary world, the need to address human and non\sphinxhyphen{}human processes together seems to be more the rule than the exception.


\paragraph{Summary}
\label{\detokenize{content/session_00/Part_00_04:Summary}}\begin{itemize}
\item {} 
There is evidence that natural fluctuations of climate may have induced the earliest human settlers of oceanic islands to head out into the ocean, and may have put later island territories under severe stress.

\item {} 
Islands are both created and destroyed by shifts and flows of the earth’s crust, forces that are largely beyond human influence.

\item {} 
The difficulty of disentangling human impacts on global climate from natural fluctuations means that working out the human contribution to these environmental changes is a complex and contested process.

\item {} 
While territories may appear stable or fixed, our planet is constantly in the making, all the way down to the earth’s crust and the molten rock beneath it.

\end{itemize}


\subsection{5 Conclusion}
\label{\detokenize{content/session_00/Part_00_05:5-Conclusion}}\label{\detokenize{content/session_00/Part_00_05::doc}}
The issue of climate change draws attention to the power of human activity to transform the planet in its entirety, and it is brought into sharp focus by the predicament of low\sphinxhyphen{}lying islands like Tuvalu. As we have seen in this course, the issue of rising sea level and other potential impacts of changing global climate also point to the transformations in the physical world that occur even without human influence. Oceanic islands provide a particularly cogent reminder that the living things with
which we share our world, the patterns of the weather, and even the earth beneath our feet, shift and change of their own accord. Faced with a world in which there is instability and movement all around, and deep beneath our feet, we might easily lose our bearings completely. Here, the concepts of territory and flow and a sense of their dynamic interplay are useful, for they help us to ‘get a fix’ on a world which is always in the making.

Life, weather and geological processes are all dynamic forces that play a part in the forming of islands, and continue to contribute to their ongoing transformation. Similar dynamics are also at work on larger land masses, but because oceanic islands are encircled by sea and often far from other lands, they are especially useful for drawing out the different forces and elements that work together, or sometimes against each other, in the making of the world.

Therefore, when we come to think about the ways in which human activities are transforming islands and the world around them, we must also take into consideration the non\sphinxhyphen{}human processes and activities that are inevitably entangled with the things that we do. This course has introduced the concept of territory as a way of thinking through the processes by which different elements weave themselves together to form a coherent and integrated whole. Looking at islands as examples of territories, we
have seen some of the many ways that human and non\sphinxhyphen{}human elements combine forces in the making of places with a recognisable identity \textendash{} such as Tuvalu \textendash{} that people identify with and refer to as home.

The course also introduced the concept of flow, which offers a way of thinking about how both human and non\sphinxhyphen{}human elements circulate through the world, moving within and between territories, in a manner that keeps these territories constantly in touch with the world around them. Territories and flows work together in diverse and dynamic ways to make the world, yet we have seen that they can also interact in unsettling ways to unmake the world. Climate is one such interplay of territory and flow:
a vitally important one for human and non\sphinxhyphen{}human life. And variations or changes in climate can be both an opportunity and a challenge for human beings and other living things.

The issue of human\sphinxhyphen{}induced climate change highlights the flows that connect people’s lives on one part of the planet with the lives of others elsewhere, often half the world away. How we respond to the threat of rising sea level and other manifestations of changing climate is not only a matter of acknowledging that there is a serious problem and working out how best to alleviate it; it is a matter of recognising that we are always already entangled in the world \textendash{} the physical world as much as
the social world \textendash{} and that whatever response we make comes from being caught up in the thick of things.

An appreciation of the dynamic interaction of territories and flows can help to make us aware of the depth of this entanglement. It reminds us that things might have come together differently; the world \sphinxstyleemphasis{could} have been otherwise and, because it is a dynamic planet, the world \sphinxstyleemphasis{will} be otherwise. Considering territories and flows has also shown that there are things that we can influence or redirect, and things that are beyond our influence. There are times when the important or decisive
transformations wrought on an island territory are not of human making, as in the case of geological events such as volcanoes or colonisation by biological life. At other moments, it is human activities that have made the crucial difference, such as the discovery and settling of an island or the forming of an independent nation state.

As you will recall from Section 2, writers such as Young (2003) and Allen (2006) speak of a form of responsibility that takes account of the way that people’s actions in one part of the world can influence the lives of distant others. Because these actions and their impacts are often small, subtle and difficult to track, this sort of shared responsibility can be more complicated than directly attributing guilt or blame (Allen, 2006). This unit demonstrates that there is an added complication of
trying to disentangle the many, small cumulative actions of human beings from the changes wrought by other, non\sphinxhyphen{}human forces and processes. This tends to make the apportioning of responsibility even more challenging, as is evidenced in the debates about climate change we encountered in Section 4.3. At the same time, the potentially momentous impact of global environmental change on places like Tuvalu is a compelling reason to not shy away from such challenges.

As we began to see in this course, there are options about the way we organise our interactions with the world. International agreements, like the Kyoto Protocol, suggest that major shifts are possible and, in the subsequent chapters of this book, you will encounter other possibilities for reordering the way that certain flows and territories work.



\renewcommand{\indexname}{Index}
\printindex
\end{document}